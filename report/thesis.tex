% do not change these two lines (this is a hard requirement
% there is one exception: you might replace oneside by twoside in case you deliver 
% the printed version in the accordant format
\documentclass[11pt,titlepage,oneside,openany]{book}
\usepackage{times}


\usepackage{graphicx}
\usepackage{latexsym}
\usepackage{amsmath}
\usepackage{amssymb}

\usepackage{ntheorem}

% \usepackage{paralist}
\usepackage{tabularx}

% this packaes are useful for nice algorithms
\usepackage{algorithm}
\usepackage{algorithmic}

% well, when your work is concerned with definitions, proposition and so on, we suggest this
% feel free to add Corrolary, Theorem or whatever you need
\newtheorem{definition}{Definition}
\newtheorem{proposition}{Proposition}


% its always useful to have some shortcuts (some are specific for algorithms
% if you do not like your formating you can change it here (instead of scanning through the whole text)
\renewcommand{\algorithmiccomment}[1]{\ensuremath{\rhd} \textit{#1}}
\def\MYCALL#1#2{{\small\textsc{#1}}(\textup{#2})}
\def\MYSET#1{\scshape{#1}}
\def\MYAND{\textbf{ and }}
\def\MYOR{\textbf{ or }}
\def\MYNOT{\textbf{ not }}
\def\MYTHROW{\textbf{ throw }}
\def\MYBREAK{\textbf{break }}
\def\MYEXCEPT#1{\scshape{#1}}
\def\MYTO{\textbf{ to }}
\def\MYNIL{\textsc{Nil}}
\def\MYUNKNOWN{ unknown }
% simple stuff (not all of this is used in this examples thesis
\def\INT{{\mathcal I}} % interpretation
\def\ONT{{\mathcal O}} % ontology
\def\SEM{{\mathcal S}} % alignment semantic
\def\ALI{{\mathcal A}} % alignment
\def\USE{{\mathcal U}} % set of unsatisfiable entities
\def\CON{{\mathcal C}} % conflict set
\def\DIA{\Delta} % diagnosis
% mups and mips
\def\MUP{{\mathcal M}} % ontology
\def\MIP{{\mathcal M}} % ontology
% distributed and local entities
\newcommand{\cc}[2]{\mathit{#1}\hspace{-1pt} \# \hspace{-1pt} \mathit{#2}}
\newcommand{\cx}[1]{\mathit{#1}}
% complex stuff
\def\MER#1#2#3#4{#1 \cup_{#3}^{#2} #4} % merged ontology
\def\MUPALL#1#2#3#4#5{\textit{MUPS}_{#1}\left(#2, #3, #4, #5\right)} % the set of all mups for some concept
\def\MIPALL#1#2{\textit{MIPS}_{#1}\left(#2\right)} % the set of all mips





\begin{document}

\pagenumbering{roman}
% lets go for the title page, something like this should be okay
\begin{titlepage}
	\vspace*{2cm}
  \begin{center}
   {\Large The Extremes of Good and Evil\\}
   \vspace{2cm} 
   {Master Thesis\\}
   \vspace{2cm}
   {presented by\\
    Earl Hickey \\
    Matriculation Number 9083894\\
   }
   \vspace{1cm} 
   {submitted to the\\
    Data and Web Science Group\\
    Prof.\ Dr.\ Right Name Here\\
    University of Mannheim\\} \vspace{2cm}
   {August 2014}
  \end{center}
\end{titlepage} 

% no lets make some add some table of contents
\tableofcontents
\newpage

\listofalgorithms

\listoffigures

\listoftables

% evntuelly you might add something like this
% \listtheorems{definition}
% \listtheorems{proposition}

\newpage


% okay, start new numbering ... here is where it really starts
\pagenumbering{arabic}

\chapter{Phase I - Data Translation}
\label{cha:data-translation}



If you cite something, do it in the following way. 
\begin{itemize}
	\item Conference Proceedings: This problem is typically addressed by approaches for selecting the optimal matcher based on the nature of the matching task and the known characteristics of the different matching systems. Such an approach is described in \cite{mochol08matcher}.
	\item Journal Article: S-Match, described in \cite{giunchiglia2008semanticmatching}, employs sound and complete reasoning procedures. Nevertheless, the underlying semantic is restricted to propositional logic due to the fact that ontologies are interpreted as tree-like structures.
	\item Book: According to Euzenat and Shvaiko \cite{euzenat07matcherbook}, we define a correspondence as follows.
\end{itemize}
These are some randomly chosen examples from other works. Take a look at the end of this thesis so see how the bibliography is included.

 
\section{Use Case \& Data Profiling}


\section{Consolidated Schema \& Transformations}


\chapter{Phase II - Identity Resolution}
\label{cha:identity-resolution}

\section{Gold Standard}
\label{sec:gold-standard-IR}

In order to create the gold standard we ran initial identity resolutions with cheap matching rules. With a threshold of 0.3 we use Jaccard 3 Grams, Levensthein Similarity and a combined matching rule (check IDs)!! File Gold Standard integration.
The results were then combined into one file which for each individual correspondence outline the similarity calculated by every similarity measure as well as the company names.
Correspondences were labeld as certain matches if at least one of the similarity scores had a high matching threshold > 0.95 and an actual match was present.
Correspondences were labeled as fuzzy if the similarity measures did not agree on a rating which was indicated by the average similarity. 
Correspondecnes with low average similarity were labeled as obvious non matches after verifying that they indeed do not match.
Afterwards a random sample out of every category was drawn so that the distribution 20/30/50 distribution for the gold standard was being met.
The data was then split into train and test set with an sklearn python script, with a test size of … and stratified on the gold standard category.

--Note on balance of gold standard

\section{Matching Rules}
\label{sec:good}

\section{Blockers}
\label{sec:good}

\section{Analysis of Errors}
\label{sec:evil}


\chapter{Phase III - Data Fusion}
\label{cha:data-fusion}


\begin{table}[h]

\begin{center}
\begin{tabular*}{\textwidth}{@{\extracolsep{\fill}}>{\scriptsize}l|>{\scriptsize}c>{\scriptsize}c>{\scriptsize}c|>{\scriptsize}c>{\scriptsize}c>{\scriptsize}c>{\scriptsize}c} 
& \multicolumn{3}{>{\scriptsize}c|}{Baselines} & \multicolumn{4}{>{\scriptsize}c}{Decision Tree} \\\hline
Ontology & M(edian) & G(ood) & E(vil) & results & $\Delta$-M & $\Delta$-G & $\Delta$-E \\\hline\hline
\#301 & 0.825 & 0.877 & 0.877 & 0.855 & +0.030 & -0.022 & -0.022 \\\hline
\#302 & 0.709 & 0.753 & 0.753 & 0.753 & +0.044 & +0.000 & +0.000 \\\hline
\#303 & 0.804 & 0.860 & 0.891 & 0.816 & +0.012 & -0.044 & -0.075 \\\hline
\#304 & 0.940 & 0.961 & 0.961 & 0.967 & +0.027 & +0.006 & +0.006 \\\hline
\bfseries Average & \bfseries 0.820 & \bfseries 0.863 & \bfseries 0.871 & \bfseries 0.848 & \bfseries +0.028 & \bfseries -0.015 & \bfseries -0.023 

\end{tabular*}
\caption[Good vs. Evil]{Comparison between the Good and the Evil}
\label{tab:confonly}
\end{center}
\end{table}


\bibliographystyle{plain}
\bibliography{thesis-ref}


\appendix

\chapter{Program Code / Resources}
\label{cha:appendix-a}

The source code, a documentation, some usage examples, and additional test results are available at ...

They as well as a PDF version of this thesis is also contained on the CD\chapter{Phase II - Identity Resolution}
\label{cha:intro}-ROM attached to this thesis.

\chapter{Further Experimental Results}
\label{cha:appendix-b}

In the following further experimental results are ...


\newpage


\pagestyle{empty}


\section*{Ehrenw\"ortliche Erkl\"arung}
Ich versichere, dass ich die beiliegende Master-/Bachelorarbeit ohne Hilfe Dritter
und ohne Benutzung anderer als der angegebenen Quellen und Hilfsmittel
angefertigt und die den benutzten Quellen w\"ortlich oder inhaltlich
entnommenen Stellen als solche kenntlich gemacht habe. Diese Arbeit
hat in gleicher oder \"ahnlicher Form noch keiner Pr\"ufungsbeh\"orde
vorgelegen. Ich bin mir bewusst, dass eine falsche Er- kl\"arung rechtliche Folgen haben
wird.
\\
\\

\noindent
Mannheim, den 31.08.2014 \hspace{4cm} Unterschrift

\end{document}
